\documentclass[10pt, letterpaper]{article}

% Packages:
\usepackage[
    ignoreheadfoot, % set margins without considering header and footer
    top=1cm, % seperation between body and page edge from the top
    bottom=1cm, % seperation between body and page edge from the bottom
    left=1cm, % seperation between body and page edge from the left
    right=1cm, % seperation between body and page edge from the right
    footskip=0.7cm % seperation between body and footer
    % showframe % for debugging 
]{geometry} % for adjusting page geometry
\usepackage[explicit]{titlesec} % for customizing section titles
\usepackage{tabularx} % for making tables with fixed width columns
\usepackage{array} % tabularx requires this
\usepackage[dvipsnames]{xcolor} % for coloring text
\usepackage{enumitem} % for customizing lists
\usepackage{fontawesome5} % for using icons
\usepackage{amsmath} % for math
\usepackage[
    pdftitle={ <%- name %>'s CV },
    pdfauthor={<%- name %>},
    pdfcreator={LaTeX with RenderCV},
    colorlinks=true,
    urlcolor=primaryColor
]{hyperref} % for links, metadata and bookmarks
\usepackage[pscoord]{eso-pic} % for floating text on the page
\usepackage{calc} % for calculating lengths
\usepackage{bookmark} % for bookmarks
\usepackage{lastpage} % for getting the total number of pages
\usepackage{changepage} % for one column entries (adjustwidth environment)
\usepackage{paracol} % for two and three column entries
\usepackage{ifthen} % for conditional statements
\usepackage{needspace} % for avoiding page brake right after the section title
\usepackage{iftex} % check if engine is pdflatex, xetex or luatex
\usepackage{graphicx}
\usepackage{tikz}
\usepackage[none]{hyphenat} % Disable all hyphenation
\usepackage{xcolor}
\usepackage{multicol} % For multi-column adjustments
\usepackage{balance} % To balance columns on the last page

% Ensure that generate pdf is machine readable/ATS parsable:
\definecolor{primaryColor}{RGB}{0, 79, 144} % define primary color
% \definecolor{primaryColor}{RGB}{35, 41, 96} % define primary color
\definecolor{accentColor}{RGB}{46, 106, 142} % define primary color

\colorlet{subtitleColor}{accentColor!70!white}

\definecolor{baseTextColor}{RGB}{10, 25, 47} % define base text color
\ifPDFTeX
    \input{glyphtounicode}
    \pdfgentounicode=1
    \usepackage[T1]{fontenc}
    \usepackage[utf8]{inputenc}
    \usepackage{lmodern}
\fi

\usepackage[default, type1]{sourcesanspro} 

% Some settings:
\AtBeginEnvironment{adjustwidth}{\partopsep0pt} % remove space before adjustwidth environment
\setcounter{secnumdepth}{0} % no section numbering
\setlength{\parindent}{0pt} % no indentation
\setlength{\topskip}{0pt} % no top skip
\setlength{\columnsep}{0.5cm} % Set column separation
\makeatletter
\let\ps@customFooterStyle\ps@plain % Copy the plain style to customFooterStyle
\makeatother

\titleformat{\section}{
    % avoid page braking right after the section title
    \needspace{4\baselineskip}
    % make the font size of the section title 11pt and color it with the primary color
    \fontsize{11pt}{12pt}\selectfont\color{primaryColor}
}{
}{
}{
    % print bold title, give 0.15 cm space and draw a line of 0.8 pt thickness
    % from the end of the title to the end of the body
    \textbf{#1}\hspace{0.15cm}\titlerule[0.8pt]\hspace{-0.1cm}
}[] % section title formatting

\titlespacing{\section}{
    % left space:
    -1pt
}{
    % top space:
    0.3 cm
}{
    % bottom space:
    0.2 cm
} % section title spacing

% \renewcommand\labelitemi{$\vcenter{\hbox{\small$\bullet$}}$} % custom bullet points
\newenvironment{highlights}{
    \begin{itemize}[
        topsep=0.05 cm,
        parsep=0.05 cm,
        partopsep=0pt,
        itemsep=0pt,
        leftmargin=10pt
    ]
}{
    \end{itemize}
} % new environment for highlights
\renewcommand{\labelitemi}{$\bullet$}
\renewcommand{\labelitemii}{$\bullet$}

\newenvironment{highlightsforbulletentries}{
    \begin{itemize}[
        topsep=0.10 cm,
        parsep=0.10 cm,
        partopsep=0pt,
        itemsep=0pt,
        leftmargin=10pt
    ]
}{
    \end{itemize}
} % new environment for highlights for bullet entries


\newenvironment{onecolentry}{
    \Needspace{2\baselineskip} % Adjust the number as needed for your entry size
    \begin{adjustwidth}{0.5cm}{} % Add left padding of 0.5 cm
}{
    \end{adjustwidth}
} % new environment for one column entries

\newenvironment{twocolentry}[2][]{
    \Needspace{2\baselineskip} % Adjust the number as needed for your entry size
    \onecolentry
    \def\secondColumn{#2}
    \setcolumnwidth{\fill, 5 cm}
    \begin{paracol}{2}
}{
    \switchcolumn \raggedleft \secondColumn
    \end{paracol}
    \endonecolentry
} 

\newenvironment{twocolentrypro}[2][]{
    \Needspace{2\baselineskip} % Adjust the number as needed for your entry size
    \onecolentry
    \def\secondColumn{#2}
    \setcolumnwidth{\fill, 3.5 cm}
    \begin{paracol}{2}
}{
    \switchcolumn \raggedleft \secondColumn
    \end{paracol}
    \endonecolentry
} % new environment for two column entries

\newenvironment{threecolentry}[3][]{
    \Needspace{5\baselineskip} % Adjust the number as needed for your entry size
    \onecolentry
    \def\thirdColumn{#3}
    \setcolumnwidth{1 cm, \fill, 4.5 cm}
    \begin{paracol}{3}
    {\raggedright #2} \switchcolumn
}{
    \switchcolumn \raggedleft \thirdColumn
    \end{paracol}
    \endonecolentry
} % new environment for three column entries

\newenvironment{header}{
    \setlength{\topsep}{0pt}\par\kern\topsep\centering\color{primaryColor}\linespread{1.5}
}{
    \par\kern\topsep
} % new environment for the header

\newcommand{\hdivider}{{\color{primaryColor}\noindent\rule{\linewidth}{0.3mm}}\vspace{0.25cm}}
\newcommand{\subtitle}[1]{{\color{accentColor}\fontsize{10pt}{11pt}\bfseries#1}}
\newcommand{\subtitleDark}[1]{{\color{baseTextColor}\fontsize{10pt}{11pt}\bfseries#1}}
\newcommand{\subtitlethird}[1]{{\color{subtitleColor}\fontsize{9pt}{10pt}\bfseries#1}}


\newcommand{\placelastupdatedtext}{% \placetextbox{<horizontal pos>}{<vertical pos>}{<stuff>}
  \AddToShipoutPictureFG*{% Add <stuff> to current page foreground
    \put(
        \LenToUnit{\paperwidth-2 cm-0.2 cm+0.05cm},
        \LenToUnit{\paperheight-1.0 cm}
    ){\vtop{ { \null}\makebox[0pt][c]{
        \small\color{gray}\textit{<%- labels.lastUpdate %> <%- lastUpdate %>}\hspace{\widthof{<%- labels.lastUpdate %> <%- lastUpdate %>}}
    }}}%
  }%
}%

% save the original href command in a new command:
\let\hrefWithoutArrow\href

% new command for external links:
\renewcommand{\href}[2]{\hrefWithoutArrow{#1}{\ifthenelse{\equal{#2}{}}{ }{#2 }\raisebox{.15ex}{\footnotesize \faExternalLink*}}}

% Add avatar command
\newcommand{\avatar}[2][3cm]{%
  \begin{tikzpicture}
    \clip (0,0) circle (#1/2);
    \node at (0,0) {\includegraphics[width=#1,height=#1]{#2}};
  \end{tikzpicture}%
}

% Custom command for small italic text
\newcommand{\smallit}[1]{\textit{\footnotesize #1}}


% Define the \sectiontwocol command to allow sections to span both columns
\newcommand{\sectiontwocol}[1]{
    \twocolumn[\section{#1}]
}

\begin{document}
    \fontsize{9pt}{10pt}\selectfont
    \color{baseTextColor}

    % Avatar on the left, name and info on the right
    \begin{header}
        \begin{minipage}[c]{0.18\textwidth}
            \centering
            \avatar[2.5cm]{./avatar.jpg}
        \end{minipage}\hfill
        \begin{minipage}[c]{0.80\textwidth}
            {\bfseries\fontsize{25pt}{25pt}\selectfont <%- name.toUpperCase() %>}\\
            {\bfseries\fontsize{13pt}{13pt}\selectfont <%- tagline %>}\\
            
            \mbox{{\footnotesize\faMapMarker*}\hspace*{0.13cm}<%- location %>}~|~
            \mbox{{\footnotesize\faPhone*}\hspace*{0.13cm}<%- phone %>}~|~
            \mbox{\hrefWithoutArrow{mailto:<%- email %>}{{\footnotesize\faEnvelope[regular]}\hspace*{0.13cm}<%- email %>}}~|~
            \mbox{\hrefWithoutArrow{<%- linkedin %>}{{\footnotesize\faLinkedinIn}\hspace*{0.13cm}LinkedIn}}
        \end{minipage}
    \end{header}

    \vspace{0.3 cm - 0.3 cm}


    % Professional Profile Section (EJS)
    
    \sectiontwocol{<%- labels.professional_profile %>}
        
        \begin{onecolentry}
            <%- profile %>
        \end{onecolentry}
    
    % Technical Skills Section (EJS)

    \section{<%- e(labels.technical_skills) %>}
    \begin{onecolentry}
        \begin{highlights}
             <% technical_skills.forEach(function(skillCategory) { %>
                \item \textbf{<%- e(skillCategory.category) %>:} <%- e(skillCategory.skills.join(', ')) %>
            <% }); %>
        \end{highlights}
    \end{onecolentry}
    % Experience Section (EJS)
    
    \section{<%- labels.experience %>}
    <% experience.forEach(function(exp, idx) { %>
        \begin{onecolentry}
            \subtitle{<%- e(exp.title) %>} - \smallit{<%- e(exp.period) %>}
            <% if (exp.bullets) { %>
                \begin{highlights}
                    <% exp.bullets.forEach(function(bullet) { %>
                        \item <%- e(bullet) %>.
                    <% }); %>
                \end{highlights}                    
            <% } %>            
            <% if (exp.clients) { %>
                \begin{highlights}
                <% exp.clients.forEach(function(client, cidx) { %>
                    \item \textbf{<%- e(client.title) %>} 
                    <% if (client.bullets) { %>
                        \begin{highlights}
                            <% client.bullets.forEach(function(bullet) { %>
                                \item <%- e(bullet) %> 
                            <% }); %>
                        \end{highlights}
                    <% } %>
                <% }); %>
                \end{highlights}
            <% } %>
            \subtitleDark{<%- labels.technologies %>:} <%- e(inlineList(exp.technologies,', ', '.')) %>
            <% if (idx !== experience.length - 1) { %>
                \hdivider
            <% } %>
        \end{onecolentry}
    <% }); %>

    % Projects Section (EJS)

    <% projects && projects.forEach(function(project, idx) { %>
        \section{<%- labels.projects %>}
        \begin{twocolentrypro}{
            \smallit{<%- e(project.company) %>}
            }
            \subtitle{<%- e(project.name) %>} \;\textbar\;  \textit{<%- e(project.role) %>}            
        \end{twocolentrypro}
        \begin{onecolentry}
            \vspace{1\baselineskip}
            % \begin{adjustwidth}{0.4cm}{}
                <% if (project.description) { %>
                    <%- e(project.description) %>
                <% } %>
                <% if (project.technologies) { %>
                    \textbf{<%- labels.technologies %>:} <%- e(project.technologies.join(', ')) %>.
                <% } %>
                <% if (idx !== projects.length - 1) { %>
                    \vspace{1\baselineskip}
                <% } %>
            % \end{adjustwidth}
        \end{onecolentry}
       
    <% }); %>

    % Education Section (EJS)
    
    \section{<%- labels.education %>}
    <% education.forEach(function(edu, idx) { %>
        \begin{adjustwidth}{0.5 cm}{}
            \begin{twocolentry}{
                \smallit{<%- e(edu.location) %>}
            }
                \subtitle{<%- e(edu.institution) %>}\\
                \textit{<%- e(edu.degree) %>}\\
            \end{twocolentry}
            \begin{onecolentry}
                <%- e(edu.description) %>
            \end{onecolentry}
        \end{adjustwidth}
    <% }); %>

    % Languages Section (EJS)
    
    \section{<%- labels.languages2 %>}
    \begin{adjustwidth}{0.5 cm}{}
    <% languages.forEach(function(lang, idx) { %> \textbf{<%- lang.language %>}: <%- lang.level %><% if (idx !== languages.length - 1) { %>, <% } else { %>. <%} %><% }); %>
    \end{adjustwidth}

    % Hobbies Section (EJS)

    \section{<%- labels.hobbies %>}
    \begin{adjustwidth}{0.5 cm}{}
    <% hobbies.forEach(function(hobby, idx) { %>\csname <%- hobby.icon %>\endcsname~<%- hobby.name %><% if (idx !== hobbies.length - 1) { %>, <% } else { %>. <%} %><% }); %>
    \end{adjustwidth}

    % Balance columns on the last page
    \balance
\end{document}